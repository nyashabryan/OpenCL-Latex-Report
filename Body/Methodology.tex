\section{Methodology}

\subsection{Speed-Up}

\subsection{Data Transfer Overhead}
The OpenCl interface operates asynchronously. It may start to process data before all data from the CPU has been copied to the GPU. It may also allow the completed data to be read from the GPU before the whole set of data has been processed. To measure the overhead concerned with writing and reading data from the GPU it was neccessary to ensure that OpenCl waited for a single process to end before another process could be started. This was implemeted in code using the clFinish function at the end of the OpenCL\_WriteData function and the end of the OpenCL\_ReadData function. The code snippet below shows how this was implemented in C.

\begin{Cpp}
bool OpenCL_ReadData(cl_mem Buffer, size_t Size, void* Data){
	cl_int error = clEnqueueReadBuffer(
	Commands,
	Buffer,
	CL_TRUE,
	0,
	Size,
	Data,
	0, 0, 0
	);
	if(error != CL_SUCCESS){
		printf("Error: Failed to read buffer\n");
		return false;
	}
	// Wait for the commands to get serviced
	clFinish(Commands);
	return true;
}

bool OpenCL_WriteData(cl_mem Buffer, size_t Size, void* Data){
	cl_int error = clEnqueueWriteBuffer(
	Commands,
	Buffer,
	CL_TRUE,
	0,
	Size,
	Data,
	0, 0, 0
	);
	if(error != CL_SUCCESS){
		printf("Error: Failed to write buffer.\n");
		return false;
	}
	// Wait for the commands to get serviced
	clFinish(Commands);
	return true;
}
\end{Cpp}

For each NxN sized matrix, matrix multiplication was carried out on a NVDIA GeForce GT435M GPU using the OpenCL interface. The time taken for each data subset to be processed was measured and recorded. Run Time was measured using the tic() and toc() functions defined in C.The code snippet below shows how this was implemented in C.

\begin{Cpp}


void Process_OpenCL(){
	//printf("\n");
	
	OpenCL_ConstantInt(3, N);
	
	OpenCL_WriteData(A_Buffer, N*N*sizeof(float), A);
	OpenCL_WriteData(B_Buffer, N*N*sizeof(float), B);
	
	//start time
	tic();
	
	OpenCL_Run(N, LocalSize);
	
	//stop time
	printf("%lg", toc()/1e-3);
	
	OpenCL_ReadData(OutputBuffer, N*N*sizeof(float), Output_OpenCL);
}
		
\end{Cpp}

\vspace{-11mm}